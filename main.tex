%%%%%%%%%%%%%%%%%%%%%%%%%%%%%%%%%%%%%%%%%%%%%%%%%%%%%%%%%%%%%%%%%%%%%%%%%
%                                                                       %
% ustthesis_test.tex: A template file for usage with ustthesis.cls      %
%                                                                       %
%%%%%%%%%%%%%%%%%%%%%%%%%%%%%%%%%%%%%%%%%%%%%%%%%%%%%%%%%%%%%%%%%%%%%%%%%

\documentclass{ustthesis}

\usepackage{times,amsmath,epsfig}
\usepackage{graphicx,algorithm}
\usepackage[noend]{algorithmic}
\usepackage[center]{subfigure}
\usepackage{color,graphicx}
\usepackage{amsmath}
\newtheorem{proof}{Proof}
\usepackage[left=2.5cm,top=4.0cm,bottom=2.5cm, right=2.5cm]{geometry}
 
\newcommand{\red}[1]{#1}
\newcommand{\tab}[1]{\hspace{3mm}}

% \usepackage{latexsym}
    % Use the "latexsym" package when encountering the following error:
    %   ! LaTeX Error: Command \??? not provided in base LaTeX2e.
% \usepackage{epsf}
    % Use the "epsf" package for including EPS files.

%%%%%%%%%%%%%%%%%%%%%%%%%%%%%%%%%%%%%%%%%%%%%%%%%%%%%%%%%%%%%%%%%%%%%%%%%
%                                                                       %
% Preambles. DO NOT ERASE THEM. Change to suite your particular purpose.%
%                                                                       %
%%%%%%%%%%%%%%%%%%%%%%%%%%%%%%%%%%%%%%%%%%%%%%%%%%%%%%%%%%%%%%%%%%%%%%%%%

\title{Draw With Me: a facial drawing assistant for novices}  % Title of the thesis.
\author{Yingchao Cui}     % Author of the thesis.
\degree{\MPhil}             % Degree for which the thesis is.
%% or
%\degree{\PhD}              % Degree for which the thesis is.
\subject{Computer Science and Engineering}      % Subject of the Degree.
\department{Computer Science and Engineering}       % Department to which the thesis
                    % is submitted.
\advisor{Prof. Chiew-Lan Tai}     % Supervisor.
\depthead{Prof.Qiang Yang}    % department head.
\defencedate{2017}{08}{31}      % \defencedate{year}{month}{day}.

% NOTE:
%   According to the sample shown in the guidelines, page number is
%   placed below the bottom margin.  However, if the author prefers
%   the page number to be printed above the bottom margin, please
%   activate the following command.

% \PNumberAboveBottomMargin

\begin{document}

%%%%%%%%%%%%%%%%%%%%%%%%%%%%%%%%%%%%%%%%%%%%%%%%%%%%%%%%%%%%%%%%%%%%%%%%%
%                                                                       %
% Now the actual Thesis. The order of output MUST be followed:          %
%                                                                       %
%    1) TITLEPAGE                                                       %
%                                                                       %
% The \maketitle command generates the Title page as well as the        %
% Signature page.                                                       %
%                                                                       %
%%%%%%%%%%%%%%%%%%%%%%%%%%%%%%%%%%%%%%%%%%%%%%%%%%%%%%%%%%%%%%%%%%%%%%%%%

\maketitle

%%%%%%%%%%%%%%%%%%%%%%%%%%%%%%%%%%%%%%%%%%%%%%%%%%%%%%%%%%%%%%%%%%%%%%%%%
%                                                                       %
%     2) DEDICATION (Optional)                                          %
%                                                                       %
% The \dedication and \enddedication commands are optional. If          %
% specified it generates a page for dedication.                         %
%
%%%%%%%%%%%%%%%%%%%%%%%%%%%%%%%%%%%%%%%%%%%%%%%%%%%%%%%%%%%%%%%%%%%%%%%%%

% \dedication
% This is an optional section.
% \enddedication

%%%%%%%%%%%%%%%%%%%%%%%%%%%%%%%%%%%%%%%%%%%%%%%%%%%%%%%%%%%%%%%%%%%%%%%%%
%                                                                       %
%     3) ACKNOWLEDGMENTS                                                %
%                                                                       %
% \acknowledgments and \endacknowledgments defines the                  %
% Acknowledgments of the author of the Thesis.                          %
%                                                                       %
%%%%%%%%%%%%%%%%%%%%%%%%%%%%%%%%%%%%%%%%%%%%%%%%%%%%%%%%%%%%%%%%%%%%%%%%%

\acknowledgments

adviser 

thesis committee

group colleagues and friends

family 

\endacknowledgments

%%%%%%%%%%%%%%%%%%%%%%%%%%%%%%%%%%%%%%%%%%%%%%%%%%%%%%%%%%%%%%%%%%%%%%%%%
%                                                                       %
%     4) TABLE OF CONTENTS                                              %
%                                                                       %
%%%%%%%%%%%%%%%%%%%%%%%%%%%%%%%%%%%%%%%%%%%%%%%%%%%%%%%%%%%%%%%%%%%%%%%%%

\tableofcontents

%%%%%%%%%%%%%%%%%%%%%%%%%%%%%%%%%%%%%%%%%%%%%%%%%%%%%%%%%%%%%%%%%%%%%%%%%
%                                                                       %
%     5) LIST OF FIGURES (If Any)                                       %
%                                                                       %
%%%%%%%%%%%%%%%%%%%%%%%%%%%%%%%%%%%%%%%%%%%%%%%%%%%%%%%%%%%%%%%%%%%%%%%%%

\listoffigures

%%%%%%%%%%%%%%%%%%%%%%%%%%%%%%%%%%%%%%%%%%%%%%%%%%%%%%%%%%%%%%%%%%%%%%%%%
%                                                                       %
%     6) LIST OF TABLES (If Any)
%                                                                       %
%%%%%%%%%%%%%%%%%%%%%%%%%%%%%%%%%%%%%%%%%%%%%%%%%%%%%%%%%%%%%%%%%%%%%%%%%

\listoftables

%%%%%%%%%%%%%%%%%%%%%%%%%%%%%%%%%%%%%%%%%%%%%%%%%%%%%%%%%%%%%%%%%%%%%%%%%
%                                                                       %
%     7) ABSTRACT                                                       %
%                                                                       %
% \abstract and \endabstract are used to define a short Abstract for    %
% the Thesis.                                                           %
%                                                                       %
%%%%%%%%%%%%%%%%%%%%%%%%%%%%%%%%%%%%%%%%%%%%%%%%%%%%%%%%%%%%%%%%%%%%%%%%%

\begin{abstract}

%	What is the Problem / Challenge? 2-3s

%	What is the our work ? 1s
We design and implement Draw With Me, a facial drawing assisted system for 
novice users, which is able to teach them how to do better observation and 
lift their practice efficiency.

%	What is the main contribution? 1-2s

%	Evaluation and conclusion: 2-3s

\end{abstract}


%%%%%%%%%%%%%%%%%%%%%%%%%%%%%%%%%%%%%%%%%%%%%%%%%%%%%%%%%%%%%%%%%%%%%%%%%
%                                                                       %
%     8) The Actual Contents                                            %
%                                                                       %
% The command \chapters MUST BE USED to ensure that the entire content  %
% of the Thesis is double-spaced (in version 1.0).                      %
%                                                                       %
% However, in version 2.0, \chapters will be automatically added in     %
% the beginning of the first chapter.                                   %
%                                                                       %
%%%%%%%%%%%%%%%%%%%%%%%%%%%%%%%%%%%%%%%%%%%%%%%%%%%%%%%%%%%%%%%%%%%%%%%%%

%%\chapters         % Not necessary with ustthesis.cls (v2.0).

%%%%%%%%%%%%%%%%%%%%%%%%%%%%%%%%%%%%%%%%%%%%%%%%%%%%%%%%%%%%%%%%%%%%%%%%%
%                                                                       %
% Each chapter is defined via the \chapter command. The usual sectional %
% commands of LaTeX are also available.                                 %
%                                                                       %
%%%%%%%%%%%%%%%%%%%%%%%%%%%%%%%%%%%%%%%%%%%%%%%%%%%%%%%%%%%%%%%%%%%%%%%%%


\chapter{Introduction}\label{sec-introduction}

\section{Background}

\section{Motivation and Overview}

\section{Contributions}
	
		1.1 quantified drawing instructions
	
		1.2 guidance is of expert level accuracy due to facial feature reinforcement
		
				reinforcement is twofold: augmentation and refinement
	
		1.3 teaching platform, DFA driven work flow, canvas ...


\section{Outline}

%MapReduce is a successful paradigm~\cite{Dean2008}, originally proposed by Google,
%for the ease of distributed data processing on a large number of machines.
%In such a system, users specify two functions:
%(1) a {\em map} function to process an input key/value pair,
%and to generate a set of intermediate key/value pairs;
%(2) a {\em reduce} function to merge all intermediate key/value pairs associated with the same key.
%The system will automatically distribute and execute tasks on multiple machines~\cite{HADOOP, Dean2008}
%or multiple CPUs in a single machine~\cite{Ranger2007}.
%Thus, this paradigm reduces the programming complexity so that developers can easily
%exploit the parallelism in the underlying computing resources for complex tasks.
%Encouraged by the success of CPU-based MapReduce systems, in particular, Phoenix~\cite{Ranger2007},
%we develop Mars, a MapReduce system accelerated with graphics processors, or GPUs.
%
%GPUs can be regarded as massively parallel processors with an order of magnitude higher computation
%power (in terms of number of floating point operations per second) and memory bandwidth than CPUs~\cite{Ailamaki2006}.
%%Moreover, the performance of GPUs is improving at a rate higher than Moore's law for CPUs.
%\red{Moreover, the computational performance of GPUs is improving at a rate higher than that of CPUs.}
%However, it is a challenging task to program GPUs for general-purpose computing applications, including
%those that MapReduce users are familiar with. Specifically, GPUs are traditionally designed as
%special-purpose co-processors for dedicated graphics rendering. As such, GPU cores are SIMD
%(Single-Instruction-Multiple-Data), which discourages complex control flows. Furthermore, GPU cores are virtualized, and threads are managed by the hardware. Finally,
%GPUs manage their own on-board device memory and require programmers to explicitly
%transfer data between the GPU memory and the main memory.  Additionally, the architectural details
%of GPUs vary by vendors as well as by product releases, and programmer's access to these details is limited.
%All these factors make desirable a GPGPU (General Purpose Computation on GPUs) framework
%on which users can develop correct and efficient GPU programs easily.
%
%Recently, several GPGPU programming frameworks have been introduced, such as NVIDIA CUDA~\cite{CUDA}, and AMD Brook+~\cite{BROOKPLUS}.
%\red{These frameworks significantly improve the programmability of GPUs; nevertheless, their interfaces are vendor-specific and their hardware abstractions may be unsuitable for complex applications such as those running on MapReduce.
%Therefore, we propose Mars, a MapReduce framework to ease the programming of such applications on the GPU.} Furthermore, the MapReduce framework of Mars enables the integration of GPU-accelerated code to distributed environment, like Hadoop, with the least effort.
%Our Mars system can run on multi-core CPUs (MarsCPU), on CUDA-enabled NVIDIA GPUs (MarsCUDA) or Brook+-enabled AMD GPUs (MarsBrook), or on a combination of a multi-core CPU and a GPU on a single machine.
%We further integrate Mars into Hadoop~\cite{HADOOP}, an open-source CPU-based MapReduce system on a
%network of machines, which results in MarsHadoop, where each machine can utilize its GPU with MarsCUDA or MarsBrook
%in addition to its CPU with the original Hadoop. No matter what GPU and/or CPU Mars runs on, the API (Application
%Programming Interface) to the user is the same and is similar to that of existing CPU-based MapReduce systems.
%
%Easing up GPU programming for MapReduce applications is the main goal of our work. However, a higher-level
%abstraction for programming, specifically MapReduce, comes at a price of performance.
%In particular, we identify the following three technical challenges in implementing Mars on GPUs.
%%First, since MapReduce divides up a task by data, data skews are an inherent problem in utilizing
%First, since MapReduce divides up a task by data, load imbalance is an inherent problem in utilizing
%the massive thread parallelism on the GPU, especially because GPU threads are managed by the hardware.
%\red{Second, GPUs lack efficient global synchronization mechanisms. Threads in Map or Reduce tasks are likely to have write conflicts on the output buffer.
%%Although NVIDIA GPUs provide atomic operations on the global memory, the performance of these atomic operations is so poor that using them performs worse than using the CPU \cite{ATOMIC}. Therefore, the overhead of atomic operations harm the scalability of massive GPU threads.
%While atomic operations are enabled in recent GPUs, the overhead of atomic operations would harm the scalability of massive GPU threads \cite{ATOMIC}. We consider a lock free scheme to minimize the synchronization overhead among GPU threads.}
%%It is desirable to implement a extremely low-overhead thread synchronization mechanism.
%%Therefore, it is mandatory to implement thread synchronization, and the synchronization overhead must be low so as to scale to the massive number of GPU threads.
%Third, MapReduce applications are in general data-intensive and their result sizes are data-dependent. These two characteristics pose the following requirements on programming the GPU: (a) sufficient  thread parallelism to hide the high latency and to utilize the high bandwidth of the device memory; (b) pre-allocation of output buffers in the device memory for bulk DMA transfers, as GPU memory allocation is done through the CPU before the GPU program starts.
%
%With these challenges in mind, we develop Mars for GPUs of two most
%common programming interfaces - CUDA and Brook+. We focus on
%MarsCUDA, than MarsBrook, because in our implementation
%and evaluation, CUDA was more flexible and had a higher performance than Brook+
%for MapReduce applications.
%
%In MarsCUDA, the massive thread parallelism on the GPU is well utilized
%as each thread is automatically assigned a key/value pair to work on.
%In {\em Map}, the system evenly distributes key/value
%pairs to each thread. In {\em Reduce}, we develop a simple but effective
%\red{skew handling scheme to re-distribute data evenly across all reduce tasks. }To avoid write conflicts between
%threads, we adopt a lock-free scheme that guarantees the
%correctness of parallel execution with little synchronization
%overhead. 
%
%We extend our general design of Mars from a GPU-only MapReduce framework to a MapReduce system with GPU acceleration enabled. With this extension, Mars components can work stand-alone on a single platform, e.g., MarsCUDA on a CUDA-enabled GPU or MarsCPU on a multi-core CPU, as well as to work together to utilize multiple processors, e.g., a CPU and a GPU on a single machine.
%We also integrate Mars into Hadoop to enable GPU-acceleration for individual machines in a distributed environment.
%
%\red{We evaluated the performance of Mars in comparison with its CPU-based counterparts and the native implementation without MapReduce. Our results demonstrate the effectiveness of our  GPU-oriented  optimization  strategies. On average, our MarsCUDA is 22 times faster than the CPU-based MapReduce, Phoenix~\cite{Ranger2007}, and is less than 3 times slower than the hand-tuned native CUDA implementation. Additionally, the applications developed with Mars had a code size reduction up to seven times, compared with hand-tuned native CUDA code.}
%
%\textbf{Organization:} The remainder of the thesis is organized as
%follows. We give a brief overview of GPUs, and review prior work on GPGPU and
%MapReduce in Chapter~\ref{sec-preliminaries}. We present the design
%and implementation details of Mars in Chapter~\ref{sec-design} and
%Chapter~\ref{sec-implement} respectively. We present the extension
%to multiple machines in
%Chapter~\ref{sec-beyond}. In Chapter~\ref{sec-eval}, we present our
%experimental results. Finally, we conclude in Chapter~\ref{sec-conclusion}.

\newpage

\chapter{Preliminaries and Related Work}\label{sec-preliminaries}

Drawing tutoring system

Face related applications

Interactive drawing assistance (BricoSketch)
		

Sketch based refinement

Style and Abstraction in drawing ?


%In this chapter, we first give a brief introduction on the GPU, and then review the related work on GPGPU as well as on MapReduce.
%
%
%\section{Graphics Processing Units (GPUs)}
%
%The GPU is an integral component of
%modern computers, ranging from handheld devices to high-end servers.
%GPUs are originally designed for gaming applications with fixed
%hardware pipelines for rendering. Due to the high computation power
%and rapidly improving programmability, they have recently become a
%powerful co-processor for general purpose computing~\cite{Ailamaki2006}.
%%For example, NVIDIA Tesla GPU series have been
%%adopted to high performance computers and clusters~\cite{Tesla}. For
%%more details on the GPU and its programming techniques, we refer the
%%reader to a recent book edited by Nguyen~\cite{GPUGems}.
%
%
%\begin{figure}[ht]
%\centering
%\includegraphics[width=0.65\textwidth]{figure/gpuarch.eps}
%\caption{The many-core architecture model for GPUs}
%\label{fig:manycore}
%\end{figure}
%
%
%
%%The major vendors such as NVIDIA and AMD provide similar programmable hardware pipelines, and develop similar programming frameworks.
%As shown in Figure \ref{fig:manycore}, we model the GPU as a many-core processor, which
%contains a number of SIMD multiprocessors.
%Such a many-core model is common to both AMD and NVIDIA GPUs.
%On the GPU board, there is a GRAM device memory.
%The device memory has both a high bandwidth and a high access latency.
%For example, the NVIDIA GTX280 GPU has an access latency of 400 to 600 cycles, and
%the peak memory bandwidth between the device memory and the multiprocessors
%is around 140 GB/second.
%
%Both NVIDIA CUDA and AMD Brook+ expose a parallel programming model, which does not require programmers to have knowledge of the
%graphics rendering pipeline. In this model, the system consists of a
%{\em host} (a CPU), and one or more {\em devices} (GPUs).
%GPUs are abstracted as massively data-parallel co-processors.
%CUDA and Brook+ programmers write code using C/C++ syntax with extended keywords for kernel functions,
%which are GPU programs to be executed on {\em devices}.
%
%Programming frameworks such as CUDA and Brook+ greatly improve
%the programmability of the GPU.
%However, it is still a challenging task of developing efficient GPU programs for complex applications, such as those with MapReduce, because GPUs have a special-purpose co-processor architecture and are vendor-specific on the programming frameworks for complex applications.
%\red{Although the newly introduced OpenCL~\cite{OPENCL} is an industry standard further hiding hardware details from users, Mars is at a higher level of abstraction.} OpenCL is a general-purpose programming language, with which Mars or other MapReduce frameworks can be developed.
%
%\section{GPGPU}
%
%
%
%
%GPGPU, or General Purpose Computation on GPUs, has recently emerged
%in various applications, such as linear algebra~\cite{Jiang2005,Volkov2008}, embedded system design~\cite{Feng2007}, bioinformatics~\cite{Charalambous2005}, databases~\cite{Govindaraju2006,Govindaraju2004,He2008a}, machine learning~\cite{Chu2006}, and
%distributed computing projects including Folding@home and Seti@home.
%Recently, several GPGPU languages including AMD Brook+~\cite{BROOKPLUS} (extended from Brook~\cite{Buck2004}) and NVIDIA
%CUDA~\cite{CUDA} have been proposed by GPU vendors. They usually
%expose a general-purpose, massively multi-threaded computing
%architecture and provide a programming environment similar to C/C++.
%High-level programming frameworks such as Accelerator~\cite{Tarditi2006} and RapidMind~\cite{McCool2006} are also
%developed to better facilitate GPGPU programming. These
%programming frameworks require programmers to have knowledge of specific
%programming models such as the stream programming model in Brook+~\cite{Buck2004}, or even more, knowledge of the GPU hardware
%details. By contrast, we propose to develop a MapReduce framework accelerated with
%GPUs to ease the development of a more complex class of data
%processing tasks. It provides a uniform MapReduce interface no matter whether it runs on the GPU, on the CPU, or both.
%
%We now briefly survey recent  work that developed GPGPU primitives as building blocks for various applications, in particular, those not covered in the survey by Owens et al~\cite{Owens2007}.
%Sengupta et al.~\cite{Sengupta2007} proposed the segmented scan primitive.
%He et al.~\cite{He2007} proposed a multi-pass scheme to optimize the scatter and the gather operations.
%He et al.~\cite{He2008a} further developed a small set of primitives such as prefix sum and split for relational databases.
%Additionally, CUDPP~\cite{CUDPP}, a CUDA library of data parallel primitives, was released for GPGPU computing.
%These GPU-based primitives reduce the complexity of GPU programming.
%However, even with the primitives, programmers need to write complex GPU code for data processing tasks.
%By contrast, our work further simplifies GPU programming for MapReduce programmers by providing them with a higher level and more familiar interface than the primitives.
%
%\red{This thesis focuses  on  accelerating  MapReduce  on  the  GPU,  and  provides  a  GPU-based MapReduce framework to developers. As in the original MapReduce,  it  is  up  to  developers'  choice  whether to  use  MapReduce  or  not  according  to the workload's computational characteristics. Recent studies \cite{Kerr2010}  have used data analysis techniques to categorize the computational characteristics of different workloads on the GPU. These techniques are helpful for developers to determine whether their workloads are suitable for Mars in
%specific and the GPU in general.}
%
%%Our previous study on Mars~\cite{He2008} implemented the MapReuduce framework on CUDA-enabled GPUs. This work extends the previous work in two major aspects. First, we extend the CUDA-only Mars to another large group of GPUs, so that it can run on both NVIDIA and AMD GPUs. Second, we use GPU-only Mars as a component to work with CPU-based Mars on a single machine, as well as with Hadoop in a distributed environment.
%
%\section{MapReduce}
%The MapReduce framework~\cite{Dean2008} is based on two primitives, Map and Reduce, from functional programming.
%The general form is as follows:
%\begin{quote}
%\tab \bf{Map:} $(k_1, v_1) \rightarrow list(k_2, v_2)$.
%
%\tab \bf{Reduce:} $(k_2, list(v_2)) \rightarrow list(k_3, v_3)$.
%\end{quote}
%
%
%
%The Map function takes an input key/value pair $(k_1, v_1)$ and outputs a list of intermediate key/value pairs $(k_2, v_2)$.
%The Reduce function takes all values associated with the same key and produces a list of key/value pairs.
%Programmers implement the application logic inside the Map function and the Reduce function.
%The MapReduce runtime manages the parallel execution of these two functions.
%
%
%
%The following pseudo code illustrates a program written using MapReduce.
%This program counts the number of occurrences of each word in a collection of documents~\cite{Dean2008}.
%In this program, {\em Map} and {\em Reduce} are implemented using two system-provided APIs, {\em EmitIntermediate} and {\em Emit}, respectively.
%
%\begin{quote}
%{\bf Map}(void *{\em doc}) \{ \\
%1: {\bf for} each word {\em w} in {\em doc} \\
%2: \hspace{3mm} {\bf EmitIntermediate}({\em w}, {\em 1}); // count each word once \\
%\} \\
%{\bf Reduce}(void *{\em word}, Iterator {\em values}) \{ \\
%1: int {\em result} = 0; \\
%2: {\bf for} each {\em v} in {\em values} \\
%3: \hspace{3mm} {\em result} += {\em v}; \\
%4: {\bf Emit}({\em word}, {\em result}); // output {\em word} and its count \\
%\}
%\end{quote}
%
%%\begin{quote}
%%\tab {\bf Map}(void *{\em doc}) \{ \\
%%\tab 1: \tab {\bf for} each word {\em w} in {\em doc} \\
%%\tab 2: \tab \tab {\bf EmitIntermediate}({\em w}, {\em 1}); // count each word once \\
%%\tab \} \\
%%\tab {\bf Reduce}(void *{\em word}, Iterator {\em values}) \{ \\
%%\tab 1: \tab int {\em result} = 0; \\
%%\tab 2: \tab {\bf for} each {\em v} in {\em values} \\
%%\tab 3: \tab \tab {\em result} += {\em v}; \\
%%\tab 4: \tab {\bf Emit}({\em word}, {\em result}); // output {\em word} and its count \\
%%\tab \}
%%\end{quote}
%
%There have been several MapReduce implementations since MapReduce was proposed~\cite{Dean2008}.
%Hadoop~\cite{HADOOP} is an open-source MapReduce implementation on clusters.
%Based on Hadoop, Yang et al.~\cite{Yang2007} added the merge operation to MapReduce for the ease of relational databases operations.
%Phoenix~\cite{Ranger2007} is an efficient MapReduce runtime system on multi-core CPUs.
%Kruijf et al.~\cite{Kruijf2007} developed MapReduce on the Cell BE.
%Yeung et al.~\cite{Yeung2008} implemented an FPGA-based MapReduce system.
%
%\red{Let us briefly introduce the implementation of Phoenix~\cite{Ranger2007}. A key component in Phoenix is a scheduler, for buffer management and task distribution. The scheduler starts the Map stage by evenly dividing the input buffer into small chunks, and assigns the chunks to map workers dynamically. Each map worker runs in a CPU thread. The Reduce stage does not start until all Map tasks are done. The scheduler groups the intermediate output from the Map stage by key, and a Reduce worker processes values associated with the same key. Reduce tasks are assigned to workers dynamically. Each reduce worker maintains a static array for outputting results, and sorts this static array using insertion sort. Finally, the scheduler merges all output arrays of reduce workers into a single one. Because the output data size is not known in advance, the scheduler first allocates buffers with a default small size, and then resizes the buffer as needed.}
%
%
%%Our previous work on Mars implemented MapReduce on CUDA-enabled GPUs~\cite{He2008}. 
%Catanzaro et al.~\cite{Catanzaro2008}, developed another MapReduce system on the GPU, but it required programmers to be aware of GPU hardware details, such as thread configuration and memory hierarchy.
%Finally, the Merge framework~\cite{Linderman2008}, focused on dynamically scheduling MapReduce tasks among multiple processors, dedicated to Intel products.
%By contrast, Mars hides hardware details from programmers, and works on heterogeneous GPUs, a combination of CPU and GPU on a single machine, as well as a distributed system of multiple machines.

\newpage

\chapter{Draw with Me} \label{sec-mainwork}

		In this chapter, we introduce our facial drawing assistant named  \textit{Draw with Me}  in the following four sections. First we describe the system UI and basic operations to show how it works for users. Second we explain visual guides provided by the system, which is the main technical contribution of our work, and how to generate such accurate guidance by introducing our feature enhancement algorithms. In the third section we talk about the canvas, which mimics traditional pencil sketching and provided more efficient operations for learners through practicing their drawings. In the fourth section we explain how to coordinate the interactions during the teaching and practicing process, in a manner of state transformation with the help of the classic DFA (deterministic finite automaton) model.

\section{User Interface}

		\textit{Draw with Me} consists of three basic logical components, the tutor, the canvas and the DFA guided interaction. In the following, we will elaborate them one by one. 
	
		\begin{figure}[htbp]
			\centering
				\includegraphics[width=0.70\textwidth]{D:/documents/ust2013/UI_1.JPG}
			\caption{User Interface Overview}
			\label{fig:UI}
		\end{figure}
		
		%	UI Overview
		
		Figure \ref{fig:UI} shows our main user interface is composed of two side-by-side panels. On the left is tutor panel and on the right is canvas panel. Tutor panel is used to display the selected face photo, and more importantly, to generate all kinds of visual guides on top of the photo after the learners start to draw. It is designed as our ''AI teacher'' by quantifying the human tutor's instructions into procedures the computer can understand. By providing well designed visual guides, which will be elaborated on section \ref{sec-vg}, the system does a good imitation of a human teacher. Canvas panel takes the place of physical sketching paper. It is responsible to record and display every stroke drawn by the user, either using a digital pen or a finger on a touch screen, or a mouse. The displaying strokes on the canvas have been textured in advance to mimic pencil strokes. 
		
		When we design the user interface, we hope users of our system can focus on the teaching and practicing scenario, instead of being interrupted by complicated interactions such as menu selections, button clicks etc. So the design principle is set to be as simple as possible, meanwhile simulating the physical drawing process as closely as possible. Therefore, we provide concise interface without too many buttons or sub-menus. The canvas and pencil are closest to physical reality other than operations like undo stroke for efficiency.
			
		%	UI Operations
		
		We introduce the basic operations by describing a \textbf{typical} scenario of using our system. After starting up \textit{Draw with Me}, users are required to select the face to draw by simply clicking ''Load Image'' on the menu bar. A navigation window will show up for users to choose a preferred portrait photo file. After that our system will pass the input photo to third-party  facial detection and landmark extraction APIs to get the raw facial feature points and do some clipping to guarantee that the facial region in the photo is fully displayed. Hereafter users can see the displayed portrait photo in the tutor panel, meanwhile the system does feature enhancement and guides generation work in the backstage. Then a six-square auxiliary box appears in the tutor panel and our system starts to teach. By following the six-square guide, it's very clear for users to see the location relations between facial facial features and the differences between square boundary lines and facial features' distribution (Outer boundary lines of six-square shows the difference to the skull while inner lines shows the differences to the facial features). We call these differences good disparities as they reveal the geometric identity of the face, on which we rely to draw the head skull. The tutor panel waits for user's drawing on the canvas before it moves on to provide another guide. Next the tutor will generate a pair of horizontal and a pair of vertical dot lines to further indicate the proportions and approximate positions of facial features: eyes, nose and the mouth. Users can draw the outline of each facial feature on the canvas based on the dot lines' auxiliary. The above two tutoring guides make users able to draw the face in a very coarse level by carefully observing locations, boundary positions as well as proportions. After that, the tutor panel will generate guides for each facial feature. Zoom options are provided by the tutor, as some facial feature details may not be easily captured in the global view. The tutoring order will be eyes, nose and mouth. Each time users learn to draw a facial feature, the tutor only highlights the region of interest (ROI) and darken the others. The drawing order is indicated by the shift of highlighted ROIs. In the whole drawing process, users can add and undo strokes on canvas easily, without pollution caused by physical eraser. Users can save their drawings anytime reload to continue drawing easily.
		
		A collection of visual guides are displayed in Figure ?\\
		\begin{center}
			insert visual guides fig table here
			\end{center}
		
		% \subsection{Work Flow}
		Chart ? shows the typical working flow of \textit{Draw with Me}, namely from user choose a photo to draw, feature points extraction, feature enhancement to teaching and practicing interaction. \\
			\begin{center}
			insert flow chart here
			\end{center}

		According to Chart ?, some special statuses are stated here as a supplement. The first case is when users suddenly stop the tutoring process by trigger the ''Load Image'' menu again, our system will notify by a dialog window for users to choose whether continue to choose another photo or back to save the current work. Another case is we do set timers to record the practicing period, thus a notice dialog will show up when time is over to let users choose what to do, continuing awaiting for the current stage or jump to the next one. So a dead waiting will never take place. One more case is when users do not practice by following the tutor's orders, for example, maybe currently the nose guides are provided by the tutor but users are drawing in the area which is related to the mouth according to our canvas's records, \textit{Draw with Me} will also prompt a dialog window for users to choose terminate tutor's instructions to draw by themselves or continue the tutorial triggered by users correct drawing on the nose related region.

\section{Feature Enhancement} \label{sec-vg}
		In order to replace human teachers, the tutor panel provides thorough non-textual instructions by utilizing various vision cues such as highlighted regions, scaffolding lines, reckon by time and sketch animation. These visual guides cannot be so accurate and powerful without our enhanced features. Figure ? shows the raw feature points collected by our system. From which we can see that there is no feature points in the outer eyelid and flaws and inaccuracy points can also be found in the state of the art face detection techniques. To our knowledge, lack of details problem is because face detection is a general technique for global facial landmarks. Detailed eyelids are not in their current concern. Besides, mismatching problem are lying in the intrinsic of these learning based techniques, which cannot handle details the way as local optimizing algorithms do. \\
		
		\begin{center}
				insert raw feature points fig here
		\end{center}
		
		Based on the above observation, we would like to solve the facial feature points inaccuracy and deficiency (for outer eyelids) problems with the help of EZ-Sketching. EZ-Sketching is a sketch refinement tool which considers three-level optimization at the same time. We solve the feature inaccuracy problem by using EZ-Sketching's local optimization algorithm and work out the feature deficiency problem by using EZ-Sketching's local and semi-global optimizations simultaneously. Let us elaborate accordingly as follows.
		
		We modified EZ-Sketching's interface to allow directly passing strokes into the optimizer instead of from touch screen inputs. By connecting respective facial feature points into a ''feature stroke'', we can adopt local optimizer to refine the positions of mismatching points back to the underlying image features. Thus the refined feature stroke is as accurate as possible with the respective local image feature. Figure ? shows some of the refined feature points from chin contour or lower eyelids.
		
		\begin{center}
				insert fig to show local ez refined feature points here
		\end{center}
		
		To solve the feature deficiency problem, we need to augment corresponding feature points for outer eyelids. First we compute a set of artificial points for each outer eyelid as the initial feature points, based on the geometry relations illustrated in Figure ? By simply passing the initial points into EZ-Sketching's local optimizer cannot fix this problem due to the closeness and overlaps between the outer eyelids and inner ones. The same circumstance occured when we switch to other local optimization based algorithms. So we brainstorm to use EZ-Sketching local and semi-global optimiztion at the same time to optimize both outer eyelids and inner eyelids, and Figure ? shows that we achieved very exciting results.
		
		\begin{center}
				insert fig to show initial feature points for outer eyelid
		\end{center}

		\begin{center}
				insert fig to show augmented outer eyelids
		\end{center}

\section{DFA guided Interaction}

		%\subsection{Top Level Guidance}
		%
				%facial feature distribution, geometric relations, drawing order etc.
						%
		%\subsection{facial feature-wise guidance}
				%show eye, nose, mouth e.g. here 
				%
		%\subsection{Implementation}
			%
				%facial feature points collection \\
					%
				%feature enhancement \\
					%
						%a. intro 2 ez-skecthing \\
					%
						%b. augmentation, e.g. outer eye lids \\
					%
						%c. refinement, e.g. chin contour, inner eye lids, (mouth lips still not work) \\
							%
						%d. from points to curved lines \\
%
			%
							%
%\section{Drawing on Canvas}
	%
		%\subsection{Pencil Style visualization}
			%
		%\subsection{Productivity Tools}
					%save, load, withdraw user sketches
			%
					%
					%
%\section{Interaction between above two}
%
		%\subsection{Intro 2 DFA}
		%
		%\subsection{DFA flow chart in this system}



%--------------------------------------------------------------------------------------------------------------------------------------------------------------------
%In this chapter, we present our design for Mars, with emphasis on the GPU-based component.
%Our design is guided by the following three goals.
%
%\begin{enumerate}
  %\item {\em Programmability}. User code size reduction encourages programmers to use the GPU for their tasks.
  %\item {\em Flexibility}. The design should be applicable to various multi-/many-core processors, e.g., multi-core CPUs and AMD GPUs, and should be as expressive as the underlying runtime, e.g., NVIDIA CUDA, AMD Brook+ or pthreads, so that the system will work for a wide range of hardware and applications.
  %\item {\em High Performance}. The overall performance should be accelerated by the GPU effectively.
%\end{enumerate}
%
%\section{Overview}
%\red{By examining the Phoenix design, we see that there are three potential sources of overhead.  First, the tight-coupling of the Map and Reduce stages makes every application go through both stages, no matter whether they need both stages or not. Second, a dynamic thread scheduler for task assignment heavily relies on locking to implement synchronization. Third, each reduce worker may require frequent data movement for sorting the static output array, and the data movement can become a bottleneck for the overall performance. The latest paper about Phoenix also points out this problem~\cite{Yoo2009}.}
%
%
%In the Mars design, we decide to separate a MapReduce workflow into three loosely coupled stages -
%{\em Map}, {\em Group} and {\em Reduce}. The {\em Group} stage is designed to group {\em Map} output by key, which is the format for {\em Reduce} input. Our observation is that some applications need only the
%{\em Map} stage, some need both {\em Map} and {\em Group}, and some need all of the three stages.
%\red{The Group stage is the same as running Reduce with the identity function in the original MapReduce system \cite{Dean2008}.
%Our purpose of providing an explicit Group stage is to allow a MapReduce application with high flexibility to customize its workflow,
%and to avoid the overhead of entering unnecessary stages. }
%No matter what configuration of the three stages is for an application, the MapReduce interface
%of Mars is unchanged - users write Map and/or Reduce functions when necessary.
%
%Moreover, we decide to use a lock-free scheme for synchronization
%and to perform in-advance buffer allocation. One reason is to
%avoid heavy overheads of locking and buffer reallocation. The
%other reason is that, current GPUs do not support locking
%or in-flight buffer reallocation. In our design, we statically
%distributes tasks to a massive number of GPU threads, so that we
%can fully utilize the parallelism of the GPU. We adopt a two-phase,
%lock-free scheme for result output. The basic idea is that, in the
%first phase, we calculate histograms on the size of output results for each
%thread, followed by a prefix sum operation on the histograms,
%so that we obtain both the exact output buffer size and the deterministic
%write position for each thread; in the second phase, we
%perform the actual computation and output. We will detail this strategy in
%Section~\ref{sec-lockfree}.
%
%\section{Data Structure} \label{sec-datastructure}
%Data structures in Mars affect the workflow, memory access patterns, and the expressiveness of the system.
%
%Since the GPU does not support dynamic memory allocation on the
%device memory during the execution of the GPU code, this limitation rules out
 %dynamic data structures such as queues and linked lists, as used in
%other MapReduce implementations. Instead, we use plain arrays as the
%main data structure in Mars. The {\em Map} stage takes input records in
%the key/value form, and outputs intermediate result records, which
%are in turn the input of the {\em Group} stage. The output of the {\em Group}
%stage is the input of the {\em Reduce} stage, and {\em Reduce} produces final output
%records. Each of the three sets - the input records, the intermediate records and the
%output records  - is stored in three arrays, i.e., the
%key array, the value array and the directory index array. The
%directory index consists of an entry of $<$key offset, key size,
%value offset, value size$>$ for each key/value pair. Given a
%directory index entry, we fetch the key or the value at the
%corresponding offset in the key array or the value array.
%
%Variable-sized types, such as strings, are supported with the directory index, since current GPUs have no such build-in types yet. If
%two key/value pairs need to be swapped, we swap their corresponding
%entries in the directory index without modifying the key and the
%value arrays.
%
%Some applications perform chained MapReduce procedures,
%where the output of one MapReduce procedure is the input of
%another one. Since the sets of
%input records, intermediate records and output records  are all
%in the three-array structure uniformly, chained MapReduce is
%supported gracefully in Mars.
%
%\section{Mars Workflow}
%Figure \ref{fig:MarsWorkflow} illustrates the workflow of Mars, assuming the data resides in the disk at the beginning.
%The Mars scheduler runs on the CPU, and schedules tasks to the GPU.
%Mars has three stages, {\em Map}, {\em Group}, and {\em Reduce}.
%
%\begin{figure}[ht]
  %\centering
  %\includegraphics[width=0.60\textwidth]{figure/Mars_workflow.eps}
  %\caption{The workflow of Mars on the GPU. }\label{fig:MarsWorkflow}
%\end{figure}
%
%Before the {\em Map} stage, Mars preprocesses on the CPU the input data from disk,
%transforming the input data to key/value pairs (input records) in main memory.
%After that, it transfers input records from the main memory to the GPU device memory.
%
%In the {\em Map} stage, {\em Map Split} dispatches
%input records to GPU threads such that the workload for all threads is even.
%Each thread executes the user-defined MapCount function to compute a local histogram on
%the number and the total size of intermediate records that {\em Map} will output.
%Then, the runtime performs a GPU-based {\em Prefix Sum} on the local histograms to
%obtain the output size and the write position for each thread.
%Finally, after the CPU allocates the output buffer in the device memory, each GPU
%thread executes the user-defined Map function and outputs results.  Since the write
%position for each thread is pre-computed and has no conflict with any other threads,
%there will be no write conflict between concurrent threads. This lock-free scheme of MapCount,
%Prefix Sum, and Map is adapted from our previous
%work~\cite{He2008a}.
%
%\red{In the {\em Group} stage, both sort-based and hash-based approaches are available for grouping records by key. However, we adopt the sort-based, because some applications require to sort all output records, and the hash-based approach has to perform additional sort within each hash bucket. }
%
%
%In the {\em Reduce} stage, {\em Reduce Split} dispatches each group of records
%with the same key to a GPU thread.
%However, it may cause load imbalance between threads, since the number of records of different groups may vary widely.
%We adopt a skew handling scheme to alleviate the load imbalance problem (Section~\ref{sec-reduce}).
%The {\em Reduce} stage then works in a lock-free scheme, similar to that in {\em Map}, to obtain the result size and the write location for each thread.
%Finally, all Reduce workers output results to a single buffer.
%
%Because these three stages are loosely coupled and not every application requires all
%stages, Mars allows users to customize the following workflows in their applications:
%\begin{itemize}
%\item MAP\_ONLY. Mars executes the {\em Map} stage only, and does not execute the {\em Group} or {\em Reduce} stage.
%\item MAP\_GROUP. Mars executes the {\em Map} and {\em Group} stages, and does not execute the {\em Reduce} stage.
%\item MAP\_GROUP\_REDUCE.  Mars executes all three stages -- {\em Map}, {\em Group}, and {\em Reduce} stages.
%\end{itemize}
%
%Because usually applications need a Map to transform input records, and a Group to
%prepare for the intermediate records to feed to Reduce, we exclude the other
%workflow configurations that skip either Map or Group in the presence of Reduce.
%
%\section{Lock-free Scheme} \label{sec-lockfree}
%
%With the array structure, we allocate the space on the device memory
%for the input data as well as for the result output before executing
%the GPU program. However, the sizes of the output from the {\em Map} and
%the {\em Reduce} stages are unknown. Moreover, write conflicts occur when
%multiple threads write results to the shared output array. To
%address these two problems, we adopt a previous lock-free output
%scheme for relational joins~\cite{He2008a}. Since the output scheme
%for the {\em Map} stage is similar to that for the {\em Reduce} stage, we
%present the scheme for the {\em Map} stage only.
%
%First, each MapCount invocation on a thread outputs three counts, i.e.,
%the number of intermediate results, the total size of intermediate keys (in bytes)
%and the total size of intermediate values (in bytes).
%Based on intermediate key sizes (or value sizes),
%Mars computes a prefix sum on these sizes and produces an array of write locations.
%A write location is the start location in the output array for a map task to write.
%Based on the number of intermediate results,
%Mars computes a prefix sum and produces an array of start locations in the output
%directory index.
%Through these prefix sums, we also know the sizes of the arrays for the intermediate
%results.
%Finally, Mars allocates arrays in the device memory with the exact sizes for
%storing the intermediate results.
%
%Second, each Map invocation on a thread outputs the intermediate key/value pairs to
%the output array.
%Since each Map has its deterministic and non-overlapping positions to write to,
%the write conflicts are avoided.
%
%
%The lock-free scheme is suitable for the massive thread parallelism on the GPU, even
%though it performs a MapCount in addition to a Map. The overhead of executing MapCount
%is application dependent, and is usually small. For example, this overhead is negligible
%in the matrix multiplication in our study, since MapCount simply emits the size without
%performing the actual multiplication. In addition, the code for MapCount function is also application dependent, while in most cases, programmers write one statement to emits output data sizes. 
%
%\red{
%\section{Rapid Group} \label{sec-rapid}
%The Group stage requires to sort intermediate records.
%However, we observe that some applications inherently have their intermediate records grouped after the Map phase, and each group has the same number of records.
%For example, [A,A,A,B,B,B,C,C,C] shows three groups with A, B, and C as the key respectively, and each group is with the same size 3.
%For such applications, Mars provides a configuration parameter for users to indicate whether the intermediate data is already grouped. The runtime automatically skips the time consuming sorting, and then dispatches each group of intermediate records with the same group size to Reduce workers.
%We name this strategy as ``Rapid Group".
%}
%
%\section{Skew handling} \label{sec-reduce}
%
%We design a skew handling scheme to distribute workloads evenly across reduce workers, where the user-defined Reduce operation is commutative and associative.
%This scheme iteratively performs the {\em Reduce} stage in the following two steps.  First, we divide the data into $M$ equal-sized chunks.  Second, we perform a reduction on each chunk. In this step, each of the $M$ threads applies the reduce function on groups of records in a single chunk.  Note, in each iteration, we perform reduction on the intermediate results with the same keys only.
%
%%This scheme starts within the {\em Group} stage, immediately after the sorting, and performs in the following three steps.  First, we divide the data into $M$ equal-sized chunks. Second, we perform a reduction on each chunk.  In this step, each of the $M$ threads applies the reduce function on groups of records in a single chunk, called {\em local reduction}.  Finally, we group the reduced intermediate results and pass them to the {\em Reduce} stage.  This skew handling scheme makes sure that {\em local reduction} in {\em Group} stage are load-balanced.  Additionally, the {\em Reduce} stage processes compact groups of records, so that it alleviates the bottleneck caused by the largest group.
%
%\section{Mars APIs}
%Mars provides a small set of APIs. Similar to the existing MapReduce
%frameworks, Mars has two kinds of APIs, the user-implemented APIs,
%which the users should implement by themselves, and the
%system-provided APIs, which the users can use as library calls.
%%We refer the readers to our conference paper~\cite{He2008} for more details about Mars APIs.
%The definitions of these APIs are in Table \ref{tb:marsapi}.
%
%\doublerulesep 0.1pt
%\begin{table}[htb]
  %\centering
 %\linespread{1.7}{ {\footnotesize
  %\caption{Mars APIs}\label{tb:marsapi}
%\vspace{2em}
  %\begin{tabular}{p{5cm}p{7.5cm}p{3.5cm}}
  %\hline
%\noalign{\smallskip}
   %\textbf{Function Name} & \textbf{Description} & \textbf{Function Type}\\
%\noalign{\smallskip}
  %\hline
   %MAP\_COUNT & It calculates the output buffer size of MAP. & User-implemented \\
   %MAP & The map function. & User-implemented \\
   %REDUCE\_COUNT & It calculates the output buffer size of REDUCE. & User-implemented \\
   %REDUCE & The reduce function. & User-implemented \\
   %EMIT\_INTERMEDIATE\_COUNT & It emits the key size and the value size in MAP\_COUNT. & System-provided \\
   %EMIT\_INTERMEDIATE & It emits the key and the value in MAP. & System-provided \\
   %EMIT\_COUNT & It emits the key size and the value size in REDUCE\_COUNT. & System-provided \\
   %EMIT & It emits the key and the value in REDUCE. & System-provided \\
%\noalign{\smallskip}
  %\hline
  %\end{tabular}
  %}}
%\end{table}



\newpage

%\input{chapter/sec-implementation}
%\input{chapter/sec-hadoop}
\chapter{Evaluation and Discussion}\label{sec-eval}
%In this chapter, we evaluate Mars on a single machine using a micro-benchmark of six applications in comparison with their CPU-based counterparts and native GPU-based implementations.
%\red{We also evaluate the performance of MarsHadoop on two connected machines.}
%
%
%\section{Experimental Setup}
%Our experiments were performed on three PCs, A, B and
 %C. Table \ref{tb:machines} shows their hardware configuration.
%Both PCs A and B run 32-bit CentOS 5.1 Linux with kernel
%2.6.18, NVIDIA CUDA 2.2, and the GPU driver 185.18.14. PC C runs
%32-bit Windows XP Pro SP3, with Brook+ 1.01.0 beta, and the GPU
%driver 8.561. All hard drives on these PCs are SATA magnetic hard
%disks with 7200 rpm. On all PCs, the main memory and the device
%memory are connected by PCI-E bus with a theoretical bandwidth of 4
%GB/sec.
%
%\doublerulesep 0.1pt
%\begin{table}[htb]
  %\centering
 %\linespread{1.7}{ {\footnotesize
  %\caption{Machine configurations}\label{tb:machines}
%\vspace{2em}
  %\begin{tabular}{p{4.5cm}p{3.5cm}p{3.5cm}p{3.5cm}}
  %\hline
%\noalign{\smallskip}
    %\textbf{Machine} & \textbf{PC A}&\textbf{PC B}&\textbf{PC C}\\
%\noalign{\smallskip}
  %\hline
    %GPU&NVIDIA GTX280& NVIDIA 8800GTX & ATI Radeon HD 3870\\
    %\# GPU core & 240 & 128 & 320 \\
    %GPU Core Clock (MHz)&602&575&775 \\
    %GPU Memory Clock (MHz)&1107&900&2250 \\
    %GPU Memory Bandwidth (GB/s)& 141.7& 86.4 &72.0\\
    %GPU Memory Capacity (MB)& 1024 & 768 & 512\\
    %CPU&Intel Core2 Quad Q6600 & Intel Core2 Quad Q6600 & Intel Pentium 4 540 \\
    %CPU Clock (MHz)&2400&2400&3200 \\
    %\# CPU core& 4 & 4 & 2\\
    %CPU Memory Capacity (MB)& 2048 & 2048 & 1024\\
   %\hline
 %\hline
%\noalign{\smallskip}
  %\end{tabular}
  %}}
%\end{table}
%
%
%
%\section{Micro-benchmark}
%
%We have implemented the following six real-world applications for
%evaluating the MapReduce framework.
%
%{\bf String Match (SM):} Each Map task searches a portion of the
%input file to check whether the target string is in the portion.
%Neither the {\em Group} nor the {\em Reduce} stage is needed.
%%The three data sets include 5 million, 10 million, and 15 million text lines to match respectively.
%
%{\bf Matrix Multiplication (MM):} Matrix multiplication is used
%intensively in analyzing the relationship of two documents. Given two
%matrices $M$ and $N$, each Map task computes multiplication for a
%row from $M$ and a column from $N$. It outputs the pair of the row
%ID and the column ID as the key and the corresponding result as the
%value. Neither the {\em Group} nor the {\em Reduce} stage is needed.
%%The three data sets include about 65 thousand, 262 thousand, and 1 million pairs of row and column multiplications respectively.
%
%
%{\bf Black-Scholes:} Black-Scholes model~\cite{Black1973} is used for calculating the price for European options according to a partial differential equation.
%For each option, a Map task computes the prices for the call and put prices of an option, and emits a structure containing the price of the option call and the price of the option put as the key, and the option id as the value. The Group stage is to rank the price of option calls. No Reduce stage is needed.
%
%
%{\bf Similarity Score (SS):} It is used in web document clustering.
%The characteristics of a document are represented using a feature
%vector of floating point numbers. Given two document features,
%$\vec{a}$ and $\vec{b}$, the similarity score between these two
%documents is defined to be $\frac{\vec{a}\cdot
%\vec{b}}{|\vec{a}|\cdot |\vec{b}|}$. SS computes the pair-wise
%similarity score for a set of documents. Each Map task computes the
%similarity score for two documents. It outputs the intermediate pair
%with the score as the key and the pair of the two document IDs as the
%value. The {\em Group} stage is required to rank the pair-wise similarity
%scores and no {\em Reduce} stage is required. %The three data sets include
%%about 130 thousand, 262 thousand, and 1 million pairs of document
%%features for calculation respectively.
%
%\red{
%{\bf Principal component analysis (PCA):} This application computes the mean vector and the covariance matrix of a set of points in the first two steps in PCA.
%The input data is stored in a matrix.
%The whole process contains two MapReduce invocations in a chain.
%The first MapReduce procedure is to find the mean for each row in the matrix, and the second is to calculate the covariance matrix.
%Neither Group nor Reduce stage is needed in the first MapReduce invocation. A Map task computes the mean for a row. In the second invocation, each Map task is to calculate the covariance of two rows.
%The Group stage is required to sort the row-pairs by row IDs.
%No Reduce phase is needed.
%}
%
%
%{\bf Monte Carlo (MC):} Monte Carlo~\cite{Boyle1977} is used to compute option pricing in financial engineering.
%The Monte Carlo numeric integration is to mathematically estimate the expectation of the price of option call.
%Each Map task is to compute the expected value of a random sample for an option, and to emit the option ID as the key, while the expected value of the random sample as the value.
%The Group stage and the Reduce stage are required to calculate the mean of all the samples for each option.
%In this application, all the options have the same number of samples, and the intermediate results are ordered by option ID already. Mars does not need to perform sorting in the Group stage.
%
%The above applications are commonly used in benchmarking MapReduce implementations in the previous studies~\cite{Chu2006, Ranger2007}. SM, MM and PCA are adopted from Phoenix suite \cite{Ranger2007},  SS is a common component in web applications,
%while BS and MC are prevalent in financial engineering, and are adopted from CUDA SDK.
%In particular, the workflow of these applications differ: SM and
%MM only have the {\em Map} stage, BS, SS and PCA have {\em Map} and {\em
%Group} stages, and MC has all the three stages. PCA has a chain of multiple MapReduce procedures, whereas other applications have
%only one MapReduce invocation.
%
%Within a single machine, we used three data sets
%for each application (S, M and L) to evaluate the scalability of the
%MapReduce framework.
%The input for SM is textual data, and is adopted from Phoenix~\cite{Ranger2007};
%The input for all the other applications contains randomly generated real numbers, ranging from zero to one.
%All these input data are stored as files in the hard disk.
%We summarize the size of input data for each application in Table \ref{tab:app}.
%
%\doublerulesep 0.1pt
%\begin{table}[htb]
  %\centering
 %\linespread{1.7}{ {\footnotesize
  %\caption{The input data sizes of the micro-benchmark}\label{tab:app}
%\vspace{2em}
  %\begin{tabular}{cp{3.0cm}p{3.0cm}p{3.0cm}}
  %\hline
%\noalign{\smallskip}
  %\textbf{Applications} &  \textbf{Small} & \textbf{Medium} & \textbf{Large}\\
%\noalign{\smallskip}
  %\hline
  %String Match  & size: 55MB & size: 105MB  & size: 160MB  \\
  %Matrix Multiplication  & 256x256  &  512x512  &  1024x1024  \\
  %Black-Scholes  & \# option: 1,000,000  & \# option: 3,000,000  & \# option: 5,000,000  \\
  %Similarity Score  & \# feature: 128, \# documents: 512  & \# feature: 128, \# documents: 1024  & \# feature: 128, \# documents: 2048  \\
  %PCA  & 1000x256  & 2000x256 &  4000x256  \\
  %Monte Carlo & \# option: 500, \# samples per option: 500  &  \# option: 500, \# samples per option: 2500  &  \# option: 500, \# samples per option: 5000 \\
 %\hline
%\noalign{\smallskip}
  %\end{tabular}
  %}}
%\end{table}
%
%
%%With the micro benchmarks, we have compared the performance and programmability of the MapReduce frameworks between the CPU and the GPU. The third party MapReduce on the CPU is the latest release of Phoenix in version 2.0.0.  As for native implementation, we have implemented the applications directly on CUDA and pthreads.  
%
%{\bf Metrics.} The wall time is the major metric for the
%performance evaluation. \red{We measure the elapsed time of each
%application from reading data from the disk till generating results
%in the main memory.} We ran each experiment five times and report the
%average value. The variation of elapsed time between runs is negligible.
%\red{The performance speedup on A over B is defined as the running time of B divided by the running time of A.
%The performance slowdown on A over B is defined as the running time of A divided by the running time of B.}
%
%We use the number of code lines written by the user as the metric on
%comparing the programmability of different MapReduce implementations
%as well as the native implementation with CUDA and Brook+. Note that we exclude comments and empty lines from the code size counting.
%
%\section{Results on a Single Machine}
%On a single machine, we have compared the performance and
%programmability of the MapReduce frameworks between the CPU and the
%GPU. \red{We have implemented the six applications on MarsCUDA, MarsCPU, and the latest release of Phoenix in version 2.0.0.}
%We have also implemented the applications directly on CUDA and pthreads respectively,
%including thread configuration, data distribution, task execution,  buffer management, and various memory optimizations.
%
%We present the results on the NVIDIA GPU in detail, and briefly
%present the results on the AMD GPU, mainly demonstrating the
%feasibility.
%\\\\
%{\em 1. Results on MarsCUDA and MarsCPU}
%
%{\bf Programmability.} Table \ref{tab:codesize} shows the comparison
%of user code size, for implementing the micro-benchmark
%with MarsCUDA, MarsCPU, Phoenix, and CUDA. By design, the code sizes
%with MarsCUDA are the same as those with MarsCPU. In general, the
%applications with MarsCPU have a smaller code size to those with
%Phoenix. Phoenix needs additional code to tune the runtime performance, for example, to setup cache sizes and data chunk size, and to specify the partition and locator functions that Mars does not require. 
%If the {\em Group} stage is required, applications like
%SS with MarsCUDA have a much smaller code size than that
%is manually written using CUDA, due to an optimized but lengthy
%group function on CUDA. \red{The user code size of MarsCUDA is up to 7 times smaller than that of the native implementation with CUDA.} 
%For Matrix Multiplication, CUDA have a smaller code size, because MarsCUDA requires additional code to prepare the input key/value pairs, while the native CUDA implementation does not. 
%
%\doublerulesep 0.1pt
%\begin{table}[htb]
  %\centering
 %\linespread{1.7}{ {\footnotesize
  %\caption{Comparison of application code size on MarsCPU, MarsCUDA, Phoenix, and  CUDA.}\label{tab:codesize}
%\vspace{2em}
  %\begin{tabular}{cccc}
  %\hline
%\noalign{\smallskip}
  %\textbf{Applications} & \textbf{Phoenix} & \textbf{MarsCUDA/MarsCPU} & \textbf{CUDA} \\
%\noalign{\smallskip}
  %\hline
    %String Match & 206 &  147 & 157 \\
  %Matrix Multiplication & 178 & 72 & 68\\
  %Black-Scholes & 199 & 147 & 721 \\
  %Similarity Score & 125 & 82 & 615 \\
  %Principal component analysis & 297 &  168 & 583 \\
  %Monte Carlo & 251 &  203& 359 \\
  %\hline
  %\end{tabular}
  %}}
%\end{table}
%
%\red{
%{\bf Overall performance on MapReduce.}  We conducted the performance evaluation of MarsCUDA and MarsCPU on PC A by comparing with Phoenix. Figure~\ref{fig:overall} shows the overall performance comparison. Both MarsCUDA and MarsCPU outperform Phoenix for the
%six applications, due to the general lock-free design of Mars.
%}
%
%\red{
%The overall
%performance of MarsCPU is generally better than
%that of Phoenix, achieving a speedup of up to 25.9x. Applications written using Phoenix always have a {\em Reduce} stage, whereas using ours they may not have.
%Phoenix maintains a global 2D array of pointers to keys array. Each keys array is in essence a contiguous buffer as a bucket for hashing, and is sorted by insertion sort when a new key arrives.
%Such design incurs two serious performance bottlenecks. First, lock-based synchronization is needed.
%Second, lots of memory buffer movements (calling {\em memmove()}) are required for insertion sort in the static array.
%In contrast, the design of Mars is lock-free and each Map task or Reduce task has deterministic output buffer sizes and writing positions,
%so neither lock nor memory management overhead would be introduced.
%In particular, BS and SS that require to rank distinct real numbers are over 10x slower on Phoenix than on MarsCPU.
%That is because Phoenix has to deploy millions of identity reduce tasks for these two applications. Our profiling results obtained from Intel VTune show that over 99\% of the total execution time of BS and SS on Phoenix is contributed to the {\em memmove()} operations in the Reduce stage.
%}
%
%\begin{figure*}[ht]
%\centerline{ \subfigure[Performance speedup on MarsCPU over Phoenix]{
  %\includegraphics[width=0.5\textwidth]{figure/MarsCPU_Phoenix.eps}
%\label{fig:marscpu_phoenix}}
%\hfill
%\subfigure[Performance speedup on MarsCUDA over MarsCPU (The entire MapReduce)]{
  %\includegraphics[width=0.5\textwidth]{figure/MarsGPU_MarsCPU.eps}
%\label{fig:marsgpu_marscpu}}
%} 
%
%\centerline{ 
%\subfigure[Performance speedup on MarsCUDA over MarsCPU (On large dataset, Map \& Reduce stages only)]{
  %\includegraphics[width=0.5\textwidth]{figure/kernel.eps}
%\label{fig:kernel}}
%}
%\caption{Performance evaluation for MarsCPU and MarsCUDA on the micro-benchmark} \label{fig:overall}
%\end{figure*}
%
%\red{
%As shown in \ref{fig:kernel}, MarsCUDA utilizes the GPU hardware to accelerate the Map and Reduce stages for the 6 applications, and outperforms MarsCPU in the two stages by 21x on average, and up to 40.9x.
%Please note that, this speedup is obtained without specific performance tuning on the GPU code, e.g., exploiting local memory.
%When it turns to the overall performance, MarsCUDA has a 10x speedup over MarsCPU for MM,  and 6x for MC, but not so impressive speedup for the other applications (Figure \ref{fig:marsgpu_marscpu}).
%}
%
%In order to figure out the source of slowdown in overall speedup, we further investigate the time breakdown of each application on the large data set for both MarsCUDA and MarsCPU. We divide the total execution time into four components,
%including the time for 1) preprocessing input data (``Preprocess"), including input file I/O, generating key/value pairs, and transfering data from main memory to device memory, 2) the
%{\em Map} stage (``Map"), 3) the {\em Group} stage (``Group"), and 4) the {\em
%Reduce} stage (``Reduce"). 
%MarsCUDA generally has a larger portion of preprocess time, involving key-val pair preparation and PCI-E I/O. In addition, the GPU-based Group stage has limited speedup over the CPU-based. We use Amdahl's law to explain this speedup involving parallel and sequential executions. Take SM for example. Although the GPU accelerates the Map phase by 20 times, the Map only takes up some 25\% in MarsCPU. According to Amdahl's law, the theoretical speedup of MarsCUDA over MarsCPU is at most 1.3. Our measurement is close to this theoretical speedup.
%The preprocess is possible to be parallelized on the multi-core CPU for MarsCUDA runtime. 
%However, we leave the parallelization decision to programmers, for the consideration that the runtime system can support more general purpose applications. 
%
%\begin{figure}[h]
%\centerline{ \subfigure[Time breakdown of
%MarsCUDA]{\includegraphics[width=0.50\linewidth]{figure/MarsGPU_Timebreakdown.eps}
%\label{fig:timebreakdowngpu}} \hfill \subfigure[Time breakdown of
%MarsCPU]{\includegraphics[width=0.50\linewidth]{figure/MarsCPU_Timebreakdown.eps}
%\label{fig:timebreakdowncpu}} } \caption{Time breakdown of MarsCUDA
%and MarsCPU on the micro-benchmark} \label{fig:timebreakdown}
%\end{figure}
%
%{\bf Scaling.} We used the clock rate scaling tool
%NVClock~\footnote{http://www.linuxhardware.org/nvclock/} to vary the
%NVIDIA GPU's core clock rate and memory clock rate, in order to
%evaluate the impact of hardware capability on MarsCUDA. Figures~\ref{fig:corerate} and~\ref{fig:memoryrate} show the
%performance result of the six applications running on MarsCUDA with
%the large data set.
%
%In general, most applications (except for SM) on MarsCUDA are
%sensitive to both core clock rate and memory clock rate.
%This result indicates that MarsCUDA can scale well as the GPU evolves.
%SM is not sensitive to the hardware scaling, since its GPU computation time is relatively small (as shown in Figure~\ref{fig:timebreakdowngpu}).
%
%\begin{figure}[ht]
%\centerline{ \subfigure[Baseline: Running at 100 MHz core clock rate. Memory clock rate: fixed to 1100 MHz. ]{\includegraphics[width=0.50\linewidth]{figure/corerate.eps}
%\label{fig:corerate}} \hfill \subfigure[Baseline: Running at 200 MHz memory clock rate. Core clock rate: fixed to 600 MHz. ]{\includegraphics[width=0.50\linewidth]{figure/memrate.eps}
%\label{fig:memoryrate}} } \caption{Varying clock rates on GTX 280.} \label{fig:freq}
%\end{figure}
%
%{\bf Comparison with native implementation.}
%Figure \ref{fig:marsgpu_cuda} shows the performance slowdown of the six applications on MarsCUDA over the native implementation, with large dataset.
%Overall, the implementation of applications based on MarsCUDA has roughly the same performance as on CUDA.
%However, MM and MC perform much poorer on MarsCUDA, mainly due to two reasons.
%One reason is rooted at the potential deficiency of MapReduce compared with a native implementation, as a previous study has already demonstrated~\cite{Ranger2007}. The other reason is that MarsCUDA does not automatically exploit the local memory to improve the temporal locality due to the lack of knowledge about specific applications.
%Similarly, Figure \ref{fig:marscpu_pthread} illustrates that applications on MarsCPU has roughly the same performance as on pthreads.
%
%{\bf Comparison with other GPU implementations.}
%There are two GPU-based MapReduce implementations in parallel to our work~\cite{Catanzaro2008,Linderman2008}, while the source code is not available in public. 
%Therefore, we are not able to conduct empirical performance study by comparing with these two implementations. 
%The peak speedup on the NVIDIA 8800 GTX GPU over on the CPU, reported by Catanzaro~\cite{Catanzaro2008}, is better than ours (150x vs 72x). However, their MapReduce runtime implementation is highly specialized for the machine learning workloads, and they compared with the sequential CPU code, while Mars is for general purpose applications, and we compared with parallel code. 
%The Merge framework~\cite{Linderman2008} reports a peak speedup of about 23x on the Intel X3000 GPU over on the CPU, while their MapReduce design is targeted on Intel's GPUs, and Mars is a general design for different many-core processors. 
%
%\begin{figure}[ht]
%\centerline{ \subfigure[MarsCUDA over CUDA.]{\includegraphics[width=0.5\linewidth]{figure/MarsGPU_CUDA.eps}
%\label{fig:marsgpu_cuda}} \hfill \subfigure[MarsCPU over pthreads.]{\includegraphics[width=0.5\linewidth]{figure/MarsCPU_pthread.eps}
%\label{fig:marscpu_pthread}}
%} 
%\centerline{ 
%\subfigure[MarsBrook over Brook+.]{\includegraphics[width=0.5\linewidth]{figure/MarsBrook_brook.eps}
%\label{fig:marsbrook_brook}} 
%}\caption{The performance slowdown of Mars over native implementations.}
%
%\label{fig:slowdown}
%\end{figure}
%
%{\em 2. Results on MarsBrook}
%
%Due to the limitation of Brook+, we have developed only two
%numerical applications (i.e., MM and SS) on MarsBrook. Table
%\ref{tab:brookcodesize} shows the code size of applications written
%in MarsBrook compared with the native implementation in Brook+. The
%result is consistent with the comparison between MarsCUDA and the
%native CUDA implementation. For example, the native implementation
%of SS has a much larger code size than that on MarsBrook,
%since SS requires a {\em Group} stage.
%
%\doublerulesep 0.1pt
%\begin{table}[htb]
  %\centering
 %\linespread{1.7}{ {\footnotesize
  %\caption{Comparison on code sizes of MM and SS using MarsBrook and Brook+.}\label{tab:brookcodesize}
%\vspace{2em}
  %\begin{tabular}{ccc}
  %\hline
%\noalign{\smallskip}
  %\textbf{Applications} & \textbf{MarsBrook} & \textbf{Brook+} \\
%\noalign{\smallskip}
  %\hline
  %MM & 66 & 93 \\
  %SS & 66 & 611  \\
  %\hline
%\noalign{\smallskip}
  %\end{tabular}
  %}}
%\end{table}
%
%Figure \ref{fig:marsbrook_brook} shows the performance slowdown of
%two applications by using MarsBrook over the native implementation.
%The implementation on top of MarsBrook is up to twice slower than
%the native implementation, which is the price to pay for the user code size reduction.
%\\\\
%{\em 3. Results on GPU/CPU co-processing of Mars}
%
%We used MarsCUDA and MarsCPU as two components in the co-processing.
%Figure \ref{fig:coprocess} shows the performance speedup of the GPU/CPU co-processing module over MarsCUDA, MarsCPU, and Phoenix, on the large dataset.
%Overall, co-processing utilizes the computation power of both the CPU and the GPU, and yields a considerable performance improvement over using MarsCPU or Phoenix on a CPU.
%However, the speedup of using co-processing over using standalone MarsCUDA is limited.
%
%The workload dispatching between MarsCUDA and MarsCPU in co-processing mainly depends on the performance comparison between the CPU processing and the GPU processing.
%The theoretical speedup of co-processing over MarsCUDA would be $(S + 1) / S$, where $S$ is the speedup of using MarsCUDA over using MarsCPU.
%For example, if the speedup $S$ is 10, then using co-processing would only outperform using standalone MarsCUDA by a factor of $\frac{10+1}{10} = 1.1$.
%Therefore, for compute-intensive applications MM, BS, SS, MC, and PCA, using co-processing cannot boost the performance considerably over using the standalone MarsCUDA.
%For SM that spends most time in preprocessing, using co-processing can hardly achieve the theoretical speedup $\frac{1+1}{1} = 2$.
%Nevertheless, applications using co-processing of MarsCUDA and MarsCPU still outperforms Phoenix with a speedup of 24 times on average, and 72 times at maximum.
%
%\begin{figure}[h]
 %\centering
 %\includegraphics[width=0.65\textwidth]{figure/coprocess.eps}
 %\caption{Performance speedup of GPU/CPU co-processing module over MarsCUDA, MarsCPU, and Phoenix.}\label{fig:coprocess}
%\end{figure}
%
%
%\section{Results on MarsHadoop}
%
%We experimented MM on MarsHadoop. We configured Hadoop on
%PC A and PC B: PC A as the master node, while PC A itself and PC B
%as slave nodes.
%
%
%Figure \ref{fig:hadoopspeedup} shows the performance speedup of
%MarsHadoop over the native Hadoop implementation on MM. As the
%matrix size varied, MarsHadoop is up to 2.8 times faster than the
%native Hadoop implementation. We further examine the time breakdown
%in the slave node, and the results are shown in Figure
%\ref{fig:hadoopmmbreakdown}. As the matrix size increases, the ratio
%for the computation time grows, indicating that Mars starts to help.
%The disk I/O is mainly due to the extra I/O caused by Hadoop
%streaming.
%
%\begin{figure}[h]
%\centerline{
%\subfigure[Performance speedup on MarsHadoop over native Hadoop.]{\includegraphics[width=0.50\linewidth]{figure/speeduphadoop.eps}
%\label{fig:hadoopspeedup}}
%\hfill
%\subfigure[Time breakdown on MarsHadoop.]{\includegraphics[width=0.5\linewidth]{figure/breakdownhadoop.eps}
%\label{fig:hadoopmmbreakdown}} } \caption{Matrix Multiplication on MarsHadoop}
%\end{figure}
%

\chapter{Conclusion and Future Work}\label{sec-conclusion}
%Graphics processors have become an efficient accelerator for
%high-performance computing. This thesis proposes Mars, which
%harnesses the GPU computation power and high memory bandwidth to
%accelerate MapReduce frameworks. Mars is applicable to run on
%NVIDIA GPUs, AMD GPUs, multi-core CPUs, and Hadoop-based distributed systems. 
%%and can be easily ported to other parallel systems, due to its lock-free design and simple array data structure. 
%%In addition, Mars is flexible enough to support a variety of applications efficiently, because of the support of customized workflow. 
%Our empirical studies show that Mars
%improves the programmability of both the NVIDIA and the AMD GPUs,
%and \red{the GPU-CPU co-processing of Mars on an NVIDIA GTX280 GPU and an Intel quad-core CPU outperformed Phoenix, the state-of-the-art MapReduce on the multi-core CPU with a speedup of up to 72 times and 24 times on average.} Additionally, integrating Mars into Hadoop enabled GPU acceleration for a network of PCs.
%
%The code and documentation of Mars can be found at
%http://www.cse.ust.hk/gpuqp/.



%%%%%%%%%%%%%%%%%%%%%%%%%%%%%%%%%%%%%%%%%%%%%%%%%%%%%%%%%%%%%%%%%%%%%%%%%
%                                                                       %
%      9) BIBLIOGRAPHY                                                  %
%                                                                       %
% This example uses bibtex to generate the required Bibliography. Refer %
% to the % the file ustthesis_test.bib for the entries of the           %
% Bibliography. Note that only the cited entries are printed.           %
%                                                                       %
% If BibTeX is not used to typeset the bibliography, replace the        %
% following line with the \begin{thebibliography} and \end{bibliography}%
% commands (the "thebibliography" environment) to process the           %
% Bibliography.                                                         %
%                                                                       %
%%%%%%%%%%%%%%%%%%%%%%%%%%%%%%%%%%%%%%%%%%%%%%%%%%%%%%%%%%%%%%%%%%%%%%%%%

%%%%%%%%%%%%%%%%%%%%%%%%%%%%%%%%%%%%%%%%%%%%%%%%%%%%%%%%%%%%%%%%%%%%%%%%%
%                                                                       %
% The recommended bibliography style is the IEEE bibliography style.    %
% "ustbib" defines the IEEE bibliography standard with the added        %
% ability of sorting the items by name of author.                       %
%                                                                       %
% If you are not using BibTeX to process your Bibliography, comment out %
% the following line.                                                   %
%                                                                       %
%%%%%%%%%%%%%%%%%%%%%%%%%%%%%%%%%%%%%%%%%%%%%%%%%%%%%%%%%%%%%%%%%%%%%%%%%

\bibliographystyle{plain}

\bibliography{ref}
% Please run "bibtex ustthesis_test" before the bibliography can be
% included.

%%%%%%%%%%%%%%%%%%%%%%%%%%%%%%%%%%%%%%%%%%%%%%%%%%%%%%%%%%%%%%%%%%%%%%%%%
%                                                                       %
%     10) APPENDIX (If Any)                                              %
%                                                                       %
% \appendix command marks the beginning of the APPENDIX part of the     %
% Thesis. The usual \chapter command is used for the different chapters %
% of the Appendix.                                                      %
%                                                                       %
%%%%%%%%%%%%%%%%%%%%%%%%%%%%%%%%%%%%%%%%%%%%%%%%%%%%%%%%%%%%%%%%%%%%%%%%%


%%%%%%%%%%%%%%%%%%%%%%%%%%%%%%%%%%%%%%%%%%%%%%%%%%%%%%%%%%%%%%%%%%%%%%%%%
%                                                                       %
%     11) BIOGRAPHY (Optional)                                          %
%                                                                       %
% \biography and \endbiography are used to define the optional          %
% Biography of the author of the Thesis.                                %
%                                                                       %
%%%%%%%%%%%%%%%%%%%%%%%%%%%%%%%%%%%%%%%%%%%%%%%%%%%%%%%%%%%%%%%%%%%%%%%%%

% \biography
% The biography of the student is ALSO optional.
% \endbiography

\end{document}
